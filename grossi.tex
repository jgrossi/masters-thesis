%% abtex2-modelo-trabalho-academico.tex, v-1.9.2 laurocesar
%% Copyright 2012-2014 by abnTeX2 group at http://abntex2.googlecode.com/ 
%%
%% This work may be distributed and/or modified under the
%% conditions of the LaTeX Project Public License, either version 1.3
%% of this license or (at your option) any later version.
%% The latest version of this license is in
%%   http://www.latex-project.org/lppl.txt
%% and version 1.3 or later is part of all distributions of LaTeX
%% version 2005/12/01 or later.
%%
%% This work has the LPPL maintenance status `maintained'.
%% 
%% The Current Maintainer of this work is the abnTeX2 team, led
%% by Lauro César Araujo. Further information are available on 
%% http://abntex2.googlecode.com/
%%
%% This work consists of the files abntex2-modelo-trabalho-academico.tex,
%% abntex2-modelo-include-comandos and abntex2-modelo-references.bib
%%

% ------------------------------------------------------------------------
% ------------------------------------------------------------------------
% abnTeX2: Modelo de Trabalho Academico (tese de doutorado, dissertacao de
% mestrado e trabalhos monograficos em geral) em conformidade com 
% ABNT NBR 14724:2011: Informacao e documentacao - Trabalhos academicos -
% Apresentacao
% ------------------------------------------------------------------------
% ------------------------------------------------------------------------

\documentclass[
	% -- opções da classe memoir --
	12pt,               % tamanho da fonte
	openright,          % capítulos começam em pág ímpar (insere página vazia caso preciso)
	twoside,            % para impressão em verso e anverso. Oposto a oneside
	a4paper,            % tamanho do papel. 
	% -- opções da classe abntex2 --
	%chapter=TITLE,     % títulos de capítulos convertidos em letras maiúsculas
	%section=TITLE,     % títulos de seções convertidos em letras maiúsculas
	%subsection=TITLE,  % títulos de subseções convertidos em letras maiúsculas
	%subsubsection=TITLE,% títulos de subsubseções convertidos em letras maiúsculas
	% -- opções do pacote babel --
	english,            % idioma adicional para hifenização
	% french,             % idioma adicional para hifenização
	% spanish,            % idioma adicional para hifenização
	brazil              % o último idioma é o principal do documento
	]{abntex2}

% ---
% Pacotes básicos 
% ---
\usepackage{lmodern}            % Usa a fonte Latin Modern          
\usepackage[T1]{fontenc}        % Selecao de codigos de fonte.
\usepackage[utf8]{inputenc}     % Codificacao do documento (conversão automática dos acentos)
\usepackage{lastpage}           % Usado pela Ficha catalográfica
\usepackage{indentfirst}        % Indenta o primeiro parágrafo de cada seção.
\usepackage{color}              % Controle das cores
\usepackage{graphicx}           % Inclusão de gráficos
\usepackage{microtype}          % para melhorias de justificação
% ---
		
% ---
% Pacotes adicionais, usados apenas no âmbito do Modelo Canônico do abnteX2
% ---
\usepackage{lipsum}             % para geração de dummy text
% ---

% ---
% Pacotes de citações
% ---
\usepackage[brazilian,hyperpageref]{backref}     % Paginas com as citações na bibl
\usepackage[alf]{abntex2cite}   % Citações padrão ABNT

% --- 
% CONFIGURAÇÕES DE PACOTES
% --- 

% ---
% Configurações do pacote backref
% Usado sem a opção hyperpageref de backref
\renewcommand{\backrefpagesname}{Citado na(s) página(s):~}
% Texto padrão antes do número das páginas
\renewcommand{\backref}{}
% Define os textos da citação
\renewcommand*{\backrefalt}[4]{
	\ifcase #1 %
		Nenhuma citação no texto.%
	\or
		Citado na página #2.%
	\else
		Citado #1 vezes nas páginas #2.%
	\fi}%
% ---

% ---
% Informações de dados para CAPA e FOLHA DE ROSTO
% ---
\titulo{Análise Comparativa de Técnicas de Extração de Metadados em Artigos Científicos sob o Ponto de Vista do Resultado Comparativo Final}
\autor{José Alberto Grossi Júnior}
\local{Belo Horizonte/MG, Brasil}
\data{2014, v-0.1.1}
\orientador{Marcello Peixoto Bax}
\instituicao{%
	Universidade Federal de Minas Gerais -- UFMG
	\par
	Escola de Ciência da Informação
	\par
	Programa de Pós-Graduação em Ciência da Informação}
\tipotrabalho{Dissertação (Mestrado)}
% O preambulo deve conter o tipo do trabalho, o objetivo, 
% o nome da instituição e a área de concentração 
\preambulo{Dissertação de mestrado apresentada à coordenação do PPGCI/UFMG com o objetivo de obtenção de título de Mestre em Ciência da Informação}
% ---


% ---
% Configurações de aparência do PDF final

% alterando o aspecto da cor azul
\definecolor{blue}{RGB}{41,5,195}

% informações do PDF
\makeatletter
\hypersetup{
		%pagebackref=true,
		pdftitle={\@title}, 
		pdfauthor={\@author},
		pdfsubject={\imprimirpreambulo},
		pdfcreator={LaTeX with abnTeX2},
		pdfkeywords={extracao}{metadados}{artigos cientificos}{tecnicas de extracao}, 
		colorlinks=true,            % false: boxed links; true: colored links
		linkcolor=blue,             % color of internal links
		citecolor=blue,             % color of links to bibliography
		filecolor=magenta,              % color of file links
		urlcolor=blue,
		bookmarksdepth=4
}
\makeatother
% --- 

% --- 
% Espaçamentos entre linhas e parágrafos 
% --- 

% O tamanho do parágrafo é dado por:
\setlength{\parindent}{1.3cm}

% Controle do espaçamento entre um parágrafo e outro:
\setlength{\parskip}{0.2cm}  % tente também \onelineskip

% ---
% compila o indice
% ---
\makeindex
% ---

% ----
% Início do documento
% ----
\begin{document}

% Retira espaço extra obsoleto entre as frases.
\frenchspacing 

% ----------------------------------------------------------
% ELEMENTOS PRÉ-TEXTUAIS
% ----------------------------------------------------------
% \pretextual

% ---
% Capa
% ---
\imprimircapa
% ---

% ---
% Folha de rosto
% (o * indica que haverá a ficha bibliográfica)
% ---
\imprimirfolhaderosto*
% ---

% ---
% Inserir a ficha bibliografica
% ---

% Isto é um exemplo de Ficha Catalográfica, ou ``Dados internacionais de
% catalogação-na-publicação''. Você pode utilizar este modelo como referência. 
% Porém, provavelmente a biblioteca da sua universidade lhe fornecerá um PDF
% com a ficha catalográfica definitiva após a defesa do trabalho. Quando estiver
% com o documento, salve-o como PDF no diretório do seu projeto e substitua todo
% o conteúdo de implementação deste arquivo pelo comando abaixo:
%
% \begin{fichacatalografica}
%     \includepdf{fig_ficha_catalografica.pdf}
% \end{fichacatalografica}
\begin{fichacatalografica}
	\vspace*{\fill}                 % Posição vertical
	\hrule                          % Linha horizontal
	\begin{center}                  % Minipage Centralizado
	\begin{minipage}[c]{12.5cm}     % Largura
	
	\imprimirautor
	
	\hspace{0.5cm} \imprimirtitulo  / \imprimirautor. --
	\imprimirlocal, \imprimirdata-
	
	\hspace{0.5cm} \pageref{LastPage} p. : il. (algumas color.) ; 30 cm.\\
	
	\hspace{0.5cm} \imprimirorientadorRotulo~\imprimirorientador\\
	
	\hspace{0.5cm}
	\parbox[t]{\textwidth}{\imprimirtipotrabalho~--~\imprimirinstituicao,
	\imprimirdata.}\\
	
	\hspace{0.5cm}
		1. Extração de informação
		2. Metadados.
		I. Artigos científicos.
		II. Universidade Federal de Minas Gerais.
		III. Escola de Ciência da Informação.
		IV. \imprimirtitulo\\            
	
	\hspace{8.75cm} CDU 02:141:005.7\\
	
	\end{minipage}
	\end{center}
	\hrule
\end{fichacatalografica}
% ---

% % ---
% % Inserir errata
% % ---
% \begin{errata}
% Elemento opcional da \citeonline[4.2.1.2]{NBR14724:2011}. Exemplo:

% \vspace{\onelineskip}

% FERRIGNO, C. R. A. \textbf{Tratamento de neoplasias ósseas apendiculares com
% reimplantação de enxerto ósseo autólogo autoclavado associado ao plasma
% rico em plaquetas}: estudo crítico na cirurgia de preservação de membro em
% cães. 2011. 128 f. Tese (Livre-Docência) - Faculdade de Medicina Veterinária e
% Zootecnia, Universidade de São Paulo, São Paulo, 2011.

% \begin{table}[htb]
% \center
% \footnotesize
% \begin{tabular}{|p{1.4cm}|p{1cm}|p{3cm}|p{3cm}|}
%   \hline
%    \textbf{Folha} & \textbf{Linha}  & \textbf{Onde se lê}  & \textbf{Leia-se}  \\
% 	\hline
% 	1 & 10 & auto-conclavo & autoconclavo\\
%    \hline
% \end{tabular}
% \end{table}

% \end{errata}
% ---

% ---
% Inserir folha de aprovação
% ---

% Isto é um exemplo de Folha de aprovação, elemento obrigatório da NBR
% 14724/2011 (seção 4.2.1.3). Você pode utilizar este modelo até a aprovação
% do trabalho. Após isso, substitua todo o conteúdo deste arquivo por uma
% imagem da página assinada pela banca com o comando abaixo:
%
% \includepdf{folhadeaprovacao_final.pdf}
%
\begin{folhadeaprovacao}

  \begin{center}
	{\ABNTEXchapterfont\large\imprimirautor}

	\vspace*{\fill}\vspace*{\fill}
	\begin{center}
	  \ABNTEXchapterfont\bfseries\Large\imprimirtitulo
	\end{center}
	\vspace*{\fill}
	
	\hspace{.45\textwidth}
	\begin{minipage}{.5\textwidth}
		\imprimirpreambulo
	\end{minipage}%
	\vspace*{\fill}
   \end{center}
		
   Trabalho aprovado. \imprimirlocal, 20 de setembro de 2014:

   \assinatura{\textbf{\imprimirorientador} \\ Orientador} 
   \assinatura{\textbf{Professor} \\ Professor Convidado 1}
   \assinatura{\textbf{Professor} \\ Professor Convidado 2}
   \assinatura{\textbf{Professor} \\ Professor Convidado 3}
   %\assinatura{\textbf{Professor} \\ Convidado 4}
	  
   \begin{center}
	\vspace*{0.5cm}
	{\large\imprimirlocal}
	\par
	{\large\imprimirdata}
	\vspace*{1cm}
  \end{center}
  
\end{folhadeaprovacao}
% ---

% ---
% Dedicatória
% ---
\begin{dedicatoria}
   \vspace*{\fill}
   \centering
   \noindent
   \textit{ Este trabalho é dedicado a todas as pessoas que desejam,\\ 
   de uma forma ou outra, superar seus objetivos pessoais.} \vspace*{\fill}
\end{dedicatoria}
% ---

% ---
% Agradecimentos
% ---
\begin{agradecimentos}
A escrever.
% Os agradecimentos principais são direcionados à Gerald Weber, Miguel Frasson,
% Leslie H. Watter, Bruno Parente Lima, Flávio de Vasconcellos Corrêa, Otavio Real
% Salvador, Renato Machnievscz\footnote{Os nomes dos integrantes do primeiro
% projeto abn\TeX\ foram extraídos de
% \url{http://codigolivre.org.br/projects/abntex/}} e todos aqueles que
% contribuíram para que a produção de trabalhos acadêmicos conforme
% as normas ABNT com \LaTeX\ fosse possível.

% Agradecimentos especiais são direcionados ao Centro de Pesquisa em Arquitetura
% da Informação\footnote{\url{http://www.cpai.unb.br/}} da Universidade de
% Brasília (CPAI), ao grupo de usuários
% \emph{latex-br}\footnote{\url{http://groups.google.com/group/latex-br}} e aos
% novos voluntários do grupo
% \emph{\abnTeX}\footnote{\url{http://groups.google.com/group/abntex2} e
% \url{http://abntex2.googlecode.com/}}~que contribuíram e que ainda
% contribuirão para a evolução do \abnTeX.

\end{agradecimentos}
% ---

% ---
% Epígrafe
% ---
% \begin{epigrafe}
% 	\vspace*{\fill}
% 	\begin{flushright}
% 		\textit{``Não vos amoldeis às estruturas deste mundo, \\
% 		mas transformai-vos pela renovação da mente, \\
% 		a fim de distinguir qual é a vontade de Deus: \\
% 		o que é bom, o que Lhe é agradável, o que é perfeito.\\
% 		(Bíblia Sagrada, Romanos 12, 2)}
% 	\end{flushright}
% \end{epigrafe}
% ---

% ---
% RESUMOS
% ---

% resumo em português
\setlength{\absparsep}{18pt} % ajusta o espaçamento dos parágrafos do resumo
\begin{resumo}
 % Segundo a \citeonline[3.1-3.2]{NBR6028:2003}, o resumo deve ressaltar o
 % objetivo, o método, os resultados e as conclusões do documento. A ordem e a extensão
 % destes itens dependem do tipo de resumo (informativo ou indicativo) e do
 % tratamento que cada item recebe no documento original. O resumo deve ser
 % precedido da referência do documento, com exceção do resumo inserido no
 % próprio documento. (\ldots) As palavras-chave devem figurar logo abaixo do
 % resumo, antecedidas da expressão Palavras-chave:, separadas entre si por
 % ponto e finalizadas também por ponto.

 % \textbf{Palavras-chaves}: latex. abntex. editoração de texto.

    A necessidade de contribuição entre a comunidade acadêmica é evidente quando da necessidade de leituras específicas de artigos científicos de autores espalhados pelo mundo. Porém, esta contribuição se dá de maneira muito pessoal, com envios manuais de artigos quando da necessidade de certos nichos acadêmicos. A dificuldade apresentada geralmente é a centralização de artigos de maneira livre e compensatória, por meio de extração automática de metadados relevantes para o catálogo destes documentos, de maneira a permitir que qualquer pesquisador, devidamente reconhecido, possa compartilhar e obter estes documentos de maneira eficaz e anônima.

    Este trabalho demonstra que as técnicas livres existentes para extração de metadados em artigos científicos não são suficientes para abranger os diversos formatos existentes de apresentação dos conteúdos, uma vez que são baseados em layout pré-definidos, sem possibilidade de expansão ou adaptação de acordo com a necessidade de certos grupos de pesquisa, cujo formato de apresentação deste tipo de documento se dá de maneira diferenciada, ou até mesmo, adaptada para seu universo de pesquisadores.

    \textbf{Palavras-chaves}: artigos científicos. extração de metadados. extração de dados em artigos.


\end{resumo}

% resumo em inglês
\begin{resumo}[Abstract]
 \begin{otherlanguage*}{english}
	
	The need of contribution existent in the academic community is focused based on the sharing of papers from authors around the world, when specific studies are needed. However, this contribution is made in a very basic and personal way, with papers sent by manual interactions from some specific research groups. The main goal is focused on the papers centralization in a free and compensatory format, by automatic relevant metadata extraction to the indexation of these documents, allowing any researcher to share and get these documents in a very effective manner. 

	This work shows how the existent metadata extraction techniques in scientific papers are not totally perfect to perform the different papers formats to present research works, once they are based on pre-defined layouts, without any change of customization according with some groups needs, because of a different presentation format, or even, adapted to your researchers' worlds.

    \textbf{Palavras-chaves}: scientific papers. metadata extraction. data extraction on scientific papers.

   % \vspace{\onelineskip}
 
   % \noindent 
   % \textbf{Key-words}: latex. abntex. text editoration.
 \end{otherlanguage*}
\end{resumo}

% resumo em francês 
% \begin{resumo}[Résumé]
%  \begin{otherlanguage*}{french}
% 	Il s'agit d'un résumé en français.
 
%    \textbf{Mots-clés}: latex. abntex. publication de textes.
%  \end{otherlanguage*}
% \end{resumo}

% % resumo em espanhol
% \begin{resumo}[Resumen]
%  \begin{otherlanguage*}{spanish}
%    Este es el resumen en español.
  
%    \textbf{Palabras clave}: latex. abntex. publicación de textos.
%  \end{otherlanguage*}
% \end{resumo}
% ---

% ---
% inserir lista de ilustrações
% ---
\pdfbookmark[0]{\listfigurename}{lof}
\listoffigures*
\cleardoublepage
% ---

% ---
% inserir lista de tabelas
% ---
\pdfbookmark[0]{\listtablename}{lot}
\listoftables*
\cleardoublepage
% ---

% ---
% inserir lista de abreviaturas e siglas
% ---
\begin{siglas}
  \item[PDF] Portable Document Format
  \item[IEEE] Institute of Electrical and Electronics Engineers
  \item[RSL] Revisão Sistemática de Literatura
\end{siglas}
% ---

% ---
% inserir lista de símbolos
% ---
% \begin{simbolos}
  % \item[$ \Gamma $] Letra grega Gama
  % \item[$ \Lambda $] Lambda
  % \item[$ \zeta $] Letra grega minúscula zeta
  % \item[$ \in $] Pertence
% \end{simbolos}
% ---

% ---
% inserir o sumario
% ---
\pdfbookmark[0]{\contentsname}{toc}
\tableofcontents*
\cleardoublepage
% ---



% ----------------------------------------------------------
% ELEMENTOS TEXTUAIS
% ----------------------------------------------------------
\textual

% ----------------------------------------------------------
% Introdução (exemplo de capítulo sem numeração, mas presente no Sumário)
% ----------------------------------------------------------
\chapter[Introdução]{Introdução}
% \addcontentsline{toc}{chapter}{Introdução}
% ----------------------------------------------------------

A necessidade de contribuição acontece de forma natural no ser humano. Os desejos em ajudar ao próximo e inclusive contribuir com alguma parte de sua formação é algo que desperta um desejo cada vez mais amplo do ponto de vista social.

Somos seres realizados pela satisfação do outro, e seu sucesso de uma forma ou outra acarreta em nosso sucesso, nossa satisfação pessoal e de certa forma profissional. Sentimos atraídos por contribuir e por compartilhar conhecimento, sendo ele umas das principais formas de realização como pessoa.

No âmbito acadêmico sempre contribuímos de uma forma ou outra com a formação de nossos colegas e parceiros de pesquisa. Esta contribuição pode ser feita com base em uma conversa informal ou até mesmo com uma ajuda em documentação ou sugestão de um texto para leitura. Esta sugestão de leitura geralmente possui um caráter muito técnico, e envolve na maioria dos casos a utilização de artigos acadêmicos.

Sabemos da existência de bases de conhecimento de maneira global e nacional, porém quando estamos falando da contribuição social, em pequena escala, interpessoal, estamos falando que contribuições físicas, com envio de sugestões de artigos para nossos amigos pesquisadores. Este envio é feito de maneira informal, e reduz tempo e aumenta consequentemente a praticidade do processo de pesquisa.

Sendo assim, esta experiência como objetivo global seria uma ferramenta poderosa de apoio à pesquisa, com pesquisadores compartilhando conhecimentos de maneira informal, anônima, e segura. Esta forma de disseminação de conhecimento traria um benefício muito grande socialmente falando, uma vez que pesquisadores iriam se unir, mesmo que virtualmente, na transmissão de conhecimento entre si próprios, fazendo do processo de pesquisa um processo mais focado e evitando o desperdício de tempo durante a fase de pesquisa e busca por conhecimento.

Para isso, a utilização de técnicas de extração de metadados deve ser utilizada de maneira eficaz, para que de maneira automática diversos artigos sejam analisados e catalogados em pequenos universos de pesquisa. Entende-se por metadados os campos básicos e necessários para que uma pesquisa por nome, por exemplo, seja feita com sucesso. Resume-se então que os metadados que esperam-se ser extraídos destes artigos são: o título do artigo, o nome e e-mail de seus autores, o resumo/abstract e as referências utilizadas.

Basicamente estes campos já permitem que uma pesquisa mais detalhada fosse feita e então o artigo localizado. Já as referências são necessárias para se fazer referências inversas de autores que publicam e são citados posteriormente, facilitando ainda mais aos pesquisadores poder, por exemplo, encontrar artigos semelhantes de uma mesma área do conhecimento.

\section{Delimitação do Problema}

De modo geral, as técnicas livre existentes para que essa extração de metadados seja feita são focadas em layouts pré-definidos, geralmente de conferências e/ou congressos internacionais, que possuem um padrão visual parecido, como é o caso do IEEE por exemplo, que segue de referência para diversos outros eventos tomando seu layout como base.

Porém, existem diversos outros eventos que possuem layouts de artigos considerados fora do padrão e, portanto, necessitam de adaptações destas técnicas para que seus trabalhos possam ser analisados e catalogados de maneira eficaz. Esta customização promoveria uma série de tentativas para verificar o melhor layout para ser utilizado em cada caso, automaticamente.

\section{Objetivo Geral}

Este trabalho possui como objetivo geral provar que as técnicas livres de extração automática de metadados em artigos científicos ainda necessitam ajustes e principalmente flexibilidade para abranger um maior números de documentos e prover então uma contribuição maior perante a comunidade acadêmica.

A necessidade de customização é uma tendência natural de qualquer ramo de atividade, de maneira a promover possibilidades de ferramentas auto-suficientes capazes de suprir as necessidades de grupos específicos de pesquisas, de eventos ou conferências, que possui padrões de apresentação de artigos personalizados e que demandam de uma análise diferenciada para que possa ser indexada e então analisada por sistemas de informação.

\subsection{Objetivos Específicos}

Com base na diferenciação de formas de apresentação de artigos científicos este trabalho tem como Objetivos Específicos identificar pontos em que técnicas de extração de metadados necessitam de adaptações flexíveis por parte da comunidade em geral, permitindo que artigos sejam analisados de maneira diferente em virtude de especificações distintas e necessidades diferenciadas de grupos de pesquisa.

Os padrões existentes no mercado são de maneira geral insuficientes para suprir as necessidades dos mais diversos eventos e/ou conferências existentes, afunilando a apenas uma pequena parcela de artigos, o que acaba gerando um desconforto e uma ineficácia das técnicas de extração de metadados existentes atualmente.

\section{Resultados Esperados}

As formas de extração de dados em artigos científicos são geralmente baseadas em layouts, ou seja, em pequenos pedaços onde certas informações devem ser informadas. Porém em virtude da grande diversidade de materiais produzidos e em função das adaptações realizadas por grupos e/ou eventos de pesquisa, este layout padrão não se mostra eficiente na abrangência total das necessidades do meio. 

Assim sendo, espera-se que certos artigos científicos não tenham seus metadados analisados de maneira eficaz por todas as técnicas livres existentes de extração de dados, uma vez que adaptações são necessárias a fim de contribuir para uma globalização destas análises, permitindo a customização então de técnicas de extração com base em mercados ou culturas diferentes.

\section{Limitações do Trabalho}

Este trabalho limita-se aos artigos científicos difundidos na comunidade científica em formato PDF, excluindo aqueles em que seu conteúdo é disponibilizados através de imagens escaneadas de documentos físicos, o que impede, em um primeiro momento, de ter os textos analisados em sua forma original, sem necessidade de processamento extra a fim de obter todo o material textual contido em tais imagens.

Além disso o trabalho pressupões que a língua inglesa seja utilizada como padrão no meio, de maneira a permitir que através de um único idioma o conhecimento seja difundido e aplicado em diversas culturas, independente de especificidades e diferenças culturais, permitindo uma difusão do conhecimento em sua mais pura forma de apresentação.

\section{Justificativa}

De maneira geral, a necessidade de centralizar estes artigos científicos existe, e a contribuição seria uma forma de aumentar cada vez mais o acesso aos materiais de pesquisa. Sendo assim, esta forma de análise e extração de matadados traria benefícios para que este repositório fosse criado, tendo então milhões e milhões de documentos em suas bases de dados.

Este trabalho é feito justamente para prover esta visão do que ainda precisa ser melhorado e pensado para que estas técnicas abranjam diversos padrões encontrados no mercado, permitindo além que usuários possam contribuir com seus próprios padrões.

\section{Estrutura}

Esta pesquisa é estruturada iniciando com uma introdução sobre o tema, a definição do problema, os objetivos gerais e específicos e sua justificativa.

O segundo capítulo tem como base o referencial teórico feito através de uma RSL (Revisão Sistemática de Literatura), tendo como base \cite{rsl-manual}, que propõe um passo-a-passo para um revisão de literatura eficaz e atingindo os resultados desejáveis pela pesquisa.

No terceiro capítulo temos a metodologia para o desenvolvimento do trabalho, as técnicas que serão aplicadas e principalmente como serão feitas. Posteriormente, no capítulo quarto temos os testes propriamente ditos, como eles foram realizados, os ambientes de teste, a seleção de artigos para testes e no quinto capítulo os resultados obtidos.

No sexto capítulo temos a conclusão, trabalhos futuros e considerações finais sobre o trabalho apresentado.

\chapter{Revisão de Literatura}

\chapter{Metodologia}

\chapter{Testes}

\section{Ambiente de Testes}

\subsection{Servidores de Teste}

\chapter{Resultados}

\chapter{Conclusão}

\section{Trabalhos Futuros}

\section{Considerações Finais}



% ---

% ----------------------------------------------------------
% ELEMENTOS PÓS-TEXTUAIS
% ----------------------------------------------------------
\postextual
% ----------------------------------------------------------

% ----------------------------------------------------------
% Referências bibliográficas
% ----------------------------------------------------------
\bibliography{references}

% ----------------------------------------------------------
% Glossário
% ----------------------------------------------------------
%
% Consulte o manual da classe abntex2 para orientações sobre o glossário.
%
%\glossary

% ----------------------------------------------------------
% Apêndices
% ----------------------------------------------------------

% ---
% Inicia os apêndices
% ---
% \begin{apendicesenv}

% % Imprime uma página indicando o início dos apêndices
% \partapendices

% % ----------------------------------------------------------
% \chapter{Quisque libero justo}
% % ----------------------------------------------------------

% \lipsum[50]

% % ----------------------------------------------------------
% \chapter{Nullam elementum urna vel imperdiet sodales elit ipsum pharetra ligula
% ac pretium ante justo a nulla curabitur tristique arcu eu metus}
% % ----------------------------------------------------------
% \lipsum[55-57]

% \end{apendicesenv}
% ---


% ----------------------------------------------------------
% Anexos
% ----------------------------------------------------------

% ---
% Inicia os anexos
% ---
% \begin{anexosenv}

% % Imprime uma página indicando o início dos anexos
% \partanexos

% % ---
% \chapter{Morbi ultrices rutrum lorem.}
% % ---
% \lipsum[30]

% % ---
% \chapter{Cras non urna sed feugiat cum sociis natoque penatibus et magnis dis
% parturient montes nascetur ridiculus mus}
% % ---

% \lipsum[31]

% % ---
% \chapter{Fusce facilisis lacinia dui}
% % ---

% \lipsum[32]

% \end{anexosenv}

%---------------------------------------------------------------------
% INDICE REMISSIVO
%---------------------------------------------------------------------
\phantompart
\printindex
%---------------------------------------------------------------------

\end{document}
