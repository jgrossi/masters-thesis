%!TEX root = ../masters.tex

\chapter{Metodologia}
\label{cha:metodology}

Este trabalho tem como metodologia uma pesquisa de caráter não-experimental e quantitativa, por se tratar de extração automática de metadados por ferramentas previamente selecionadas, tendo os resultados comparados com a extração manual do mesmo conjunto de artigos científicos.

% Explicar o passo a passo que será utilizado

Primeiramente, são filtradas as ferramentas encontradas a fim de analisar realmente as que possuem viabilidade técnica de testes dentro do objetivo da pesquisa, fazendo do projeto o mais conciso possível. Desta forma, diversos elementos serão utilizados a fim de se obter os resultados desejados.

% Explicar de modo geral como será o processo

De modo geral o procedimento de testes deste trabalho será realizado através da instalação de cada ferramenta selecionada em máquinas (servidores), permitindo que cada uma tenha seu conjunto necessário de tecnologias para seu correto funcionamento. Assim, os artigos selecionados para testes serão utilizados em cada uma destas ferramentas e seus resultados analisados, centralizados e consolidados a fim de se obter uma nota final para cada análise encontrada. O fluxo de passos necessários para a realização destes testes pode ser melhor visualizado na \autoref{fig:metodology}.

\begin{figure}
    \centering
    \caption{Processo de Metodologia}
    \label{fig:metodology}
    \includegraphics[width=0.8\linewidth]{./assets/images/metodology}
\end{figure}


\section{Escolha do Corpus}
\label{sec:corpus}

% Falar da seleção de artigos de várias áreas

Visando provar a eficiência das ferramentas - juntamente com a implementação das técnicas por elas utilizadas -, desejamos ter resultados exatos da extração de metadados, de maneira que possam ser comparados e verificados com os metadados manualmente coletados. Deste modo, foi selecionada uma série de artigos científicos das mais diversas áreas de pesquisa, de diversos eventos distintos, com padrões visuais totalmente diferentes e que são possível de ser analisados e coletados.

% Porque selecionar artigos de diferentes áreas

Em virtude da necessidade de testar estas ferramentas em um ambiente real, estes artigos selecionados foram coletados por estudantes de graduação, mestrado e doutorado de diversas instituições de ensino e áreas do conhecimento, fazendo com que esta seleção de documentos faça parte de um conjunto real e concreto, que necessita ser testada para comprovar a eficácia das ferramentas analisadas neste trabalho.

Assim, com uma seleção de artigos que faz parte do cotidiano destes estudantes podemos aferir se os resultados fornecidos pelas ferramentas podem ser considerados positivos ou negativos, com base na comparação entre os resultados automaticamente coletados e a análise visual e manual destes documentos.

% Áreas dos Artigos e Eventos

A lista destes artigos compreende um total de 100 documentos variados, com seus metadados já extraídos manualmente e todos catalogados permitindo a comparação desejada. Na \autoref{tab:papers-list} pode-se ter acesso ao número de artigos que foram selecionados por cada área do conhecimento. A relação de todos os artigos analisados pode ser obtida no \autoref{appendix:papers} deste documento.

\begin{table}
    \caption{Seleção de artigos científicos para testes (ainda em desenvolvimento)}
    \begin{center}
        \begin{tabular}{|l|l|}
            \hline
            \textbf{Área de Conhecimento} & \textbf{Total de Artigos} \\ 
            \hline
            Arquitetura e Urbanismo & 7 \\ 
            Música & 7 \\ 
            Ciência da Computação & 8 \\
            Ciência da Informação & 9 \\
            Ciências Biológicas & 7 \\
            Direito & 7 \\
            Engenharia Civil & 8 \\
            Letras & 7 \\
            Matemática Computacional & 7 \\
            Medicina & 9 \\ 
            Odontologia & 8 \\ 
            Psicologia & 9 \\
            Sociologia & 7 \\   
            \hline
            \textbf{Total} & \textbf{100} \\
            \hline 
        \end{tabular}
    \end{center}
    \label{tab:papers-list}
\end{table}

A seleção dos artigos foi feita tomando por base a máxima diferenciação visual possível na disposição dos metadados, permitindo que uma maior variedade de documentos seja analisada, aumentando ainda a confiança nos resultados obtidos.

% Artigos em Inglês, somente

Todos os artigos selecionados foram escritos na língua inglesa. Esta decisão foi tomada em virtude de, além de ser a língua inglesa a universal para disseminação de conhecimento, ela é a mais utilizada no meio acadêmico, possuindo um universo muito maior e mais rico de artigos escrito no idioma. Além disso, algumas das ferramentas e respectivas técnicas utilizadas nos testes utilizam de ''processamento de linguagem natural'' para extração dos metadados, tendo por padrão a utilização do inglês na análise dos textos dos documentos.

Em virtude destas colocações a abrangência de outros idiomas entraria em um aspecto que não é objetivo deste trabalho abordar, visto a diversificação de culturas e símbolos, fazendo com que línguas orientais, como o mandarim ou japonês por exemplo, tenham análises diferenciadas em função de suas diferenças nas formas de representação e leitura, necessitando de outras técnicas e/ou ferramentas mais direcionadas a fim de se obter resultados satisfatórios.

No que tange a escolha das ferramentas para testes foi utilizado apenas um ponto na seleção: a sua utilização por linha de comando. Embora algumas ferramentas possuem código aberto a extração de metadados faz parte de um contexto específico da aplicação, dificultando a utilização de apenas este recurso, objetivo deste trabalho. Assim, foram selecionadas para teste apenas as ferramentas que permitem o uso de sua funcionalidade de extração de metadados por linha de comando, independente da linguagem de programação ou tecnologia apresentada. 

Deste modo, as ferramentas selecionadas para teste foram: Cermine, CiteSeer, CrossRef e ParsCit, conforme detalhado na \autoref{tab:tools-consolidated}.
    

\section{Desenho do Experimento}
\label{sec:experiment-design}

Tendo selecionadas as ferramentas e os artigos que serão utilizados para os testes parte-se para a instalação adequada de cada ferramenta nos servidores de testes, juntamente com as tecnologias necessárias e as linguagens de programação solicitadas pelos seus desenvolvedores. 

Cada ferramenta será testada em separado, observando suas características particulares. Assim, cada artigo selecionado será testado para aquela ferramenta, anotando os resultados obtidos na extração dos metadados. Estes resultados serão separados por metadados, o que permitirá calcular qual a porcentagem de acerto que a ferramenta teve na extração de cada metadado analisado.

Assim, o processo será repetido para cada ferramenta e o resultado registrado, permitindo calcular sua porcentagem total de acertos de maneira simplificada. Para isso será criado um ''Quadro Comparativo', no qual serão inseridos os resultados dos testes de cada ferramenta. 

\subsection{Metadados, Pesos e Resultados}
\label{ssec:metadata-results}

% Importância de se ter pesos em função dos metadados

A extração de metadados de artigos científicos engloba um processo onde os resultados obtidos possuem características diferenciadas que podem influenciar em uma busca por artigos, feita por um pesquisador, por exemplo. Desde modo serão atribuídos pesos para cada um dos metadados analisados, de maneira a identificar os mais importantes e que podem contribuir com um número maior de resultados de busca. 

% Quais metadados são mais importantes para uma pesquisa de artigos

Alguns metadados são mais importantes que outros no que diz respeito à funcionalidade de pesquisa. Geralmente quando vamos buscar artigos, geralmente buscamos primeiro pelo título do artigo (quando procuramos por um artigo em específico) ou então pelo nome do autor (quanto procuramos artigos de um determinado autor).

Além disso, utilizamos também o título, juntamente com o resumo, para buscar por palavras chaves ou palavras que podem ser relevantes na pesquisa pelos documentos. Desta forma alguns metadados devem possuir uma relevância maior no resultado destas extrações, por serem mais importantes no ponto de vista de busca e pesquisa.

Deste modo apresentamos a \autoref{tab:metadata-weight}, que demonstra como cada metadado deve ser interpretado e qual o peso que lhe será atribuído, sendo utilizado o inteiro 1 (um) para o peso mais baixo e o 5 (cinco) para peso mais alto, sendo consequentemente o(s) mais importante(s).

% Tabela de metadados e pesos

\begin{table}
    \caption{Os metadados e seus pesos atribuídos}
    \begin{center}
        \begin{tabular}{|p{3cm}|p{8cm}|C{1cm}|}
            \hline \textbf{Metadado} & \textbf{Relevância} & \textbf{Peso} \\ 
            \hline Título & Um dos termos mais buscados quando se pesquisa um artigo & 5 \\
            \hline Autor(es) & O segundo termo mais pesquisado & 4 \\
            \hline E-mail(s) & Pouco relevante no quesito pesquisa de artigos & 1 \\
            \hline Resumo & Importante por conter palavras chaves e o resumo propriamente dito & 3 \\
            \hline Referências & Muito importante e necessário, pois será utilizada na referência inversa de autores & 4 \\
            \hline 
        \end{tabular} 
    \end{center}
    \label{tab:metadata-weight}
\end{table}

Como cada ferramenta será testada em separado, será registrado o resultado da extração de cada artigo, tendo o total da precisão calculado de acordo com a média aritmética dos resultados obtidos para aquele metadado. Por exemplo, para a Ferramenta ''A'' serão analisados 100 (cem) artigos. A porcentagem de acerto para o título de cada artigo ($P_{título1}$, $P_{título2}$, ..., $P_{títuloN}$), por exemplo, será somada e o resultado dividido pelo número de artigos - no caso 100. Assim tem-se a nota para o metadado ''Título'' para a Ferramenta ''A'' ($P_{título}$):

\begin{center}
    \begin{math}
        P_{título} = (P_{título1} + P_{título2} + P_{título3} ... + P_{título100}) / 100
        \label{math:result-by-metadata}
    \end{math}
\end{center}

De posse dos resultados para cada metadado extraído podemos comparar as ferramentas a fim de obter qual é mais adequada para cada tipo de metadado. Podemos inferir, portanto, que a ferramenta ''X'' apresento melhores resultados do que ''Y'' na extração do nome dos autores, por exemplo.

\subsection{Índice de Confiabilidade}
\label{ssec:confiability-index}

% Fórmula final para a nota final de cada técnica

Considerando que cada metadado possui um peso diferente necessitamos calcular o índice de acertos a ser utilizado em cada resultado coletado para cada ferramenta testada. Assim chegamos a uma fórmula matemática à qual chamaremos ''Índice de Confiabilidade'', que calcula o resultado obtido através dos pesos que foram atribuídos a cada campo, para cada ferramenta. Este índice é a nota final de cada ferramenta, levando em consideração todos os resultados obtidos para todos os metadados analisados.

Este índice utiliza os pesos anteriormente definidos e a precisão dos resultados obtida, de maneira a permitir chegar em uma única nota final para cada técnica aplicada.

Basicamente a fórmula é a média ponderada dos resultados alcançados na extração de cada metadado dos artigos, seguindo os pesos apresentados na \autoref{tab:metadata-weight}. Cada peso é atribuído ao resultado encontrado em cada ferramenta testada. 

A título de exemplo, após o teste de uma ferramenta, supondo que ela conseguiu extrair 87\% dos títulos de todos artigos com sucesso, sua precisão com relação ao título será 87 ($P_{título}=87$), que será multiplicado pelo peso correspondente, neste caso, o inteiro 5. Isso ocorre para todos os metadados apresentados, seguindo seus respectivos pesos. A descrição de cada variável no Índice de Confiabilidade poderá ser obtida de acordo com a \autoref{tab:confiability-index}.

\begin{center}
    $ IC_{Ferramenta X}=(5*P_{título}+4*P_{autor}+1*P_{email}+3*P_{resumo}+5*P_{referências}) / 17 $
\end{center}

\begin{table}
    \caption{Descrição de cada variável no Índice de Confiabilidade}
    \begin{center}
        \begin{tabular}{|p{3cm}|p{8cm}|}
            \hline \textbf{Variável} & \textbf{Descrição}\\ 
            \hline $P_{título}$ & Precisão na obtenção do título \\
            \hline $P_{autor}$ & Precisão na obtenção do(s) autor(es)\\
            \hline $P_{email}$ & Precisão na obtenção dos e-mails dos autores \\
            \hline $P_{resumo}$ & Precisão na obtenção do resumo \\
            \hline $P_{referências}$ & Precisão na obtenção das referências \\
            \hline 
        \end{tabular} 
    \end{center}
    \label{tab:confiability-index}
\end{table}

Assim, de posse do Índice de Confiabilidade de cada ferramenta podemos classificar cada uma com base nos seus resultados apresentados. Desta forma podemos classificar cada ferramenta seguindo os tópicos abaixo:

\begin{enumerate}
    \item \textbf{Preciso (P):} Quando o Índice de Confiabilidade é maior ou igual a 90 ($IC\geq90$).
    \item \textbf{Satisfatório (S):} Quando o Índice de Confiabilidade é maior ou igual a 80 e menor que 90 ($80 \leq IC < 90$).
    \item \textbf{Inaceitável (I):} Quando o Índice de Confiabilidade é menor que 80 ($IC < 80$).
\end{enumerate}

\section{Ambiente Tecnológico}
\label{sec:tech-environment}

Algumas ferramentas exigem um conjunto de tecnologias muito diferente das demais, como é o caso do CiteSeer (\autoref{ssec:citeseer}). Sendo assim, neste caso em específico, será utilizado um servidor apenas para os testes desta ferramenta, juntamente com as tecnologias que esta exige para um correto funcionamento. 

Outras ferramentas, por serem escritas em linguagens de programação iguais ou por terem seu conjunto tecnológico muito semelhante, serão instaladas em um único servidor, compartilhando recursos e simplificando o trabalho de configuração, visto que possuem necessidades parecidas.

Estes servidores serão criados através de máquinas virtuais, o que traz benefícios não somente de performance mas de flexibilidade quanto das tecnologias necessárias para o funcionamento de cada técnica, permitindo que os testes possam ser feitos em sistemas operacionais distintos mas utilizando dos mesmos recursos computacionais.

Deste modo, dentro de um mesmo servidor podemos criar diversos outros servidores ''virtuais'', com características diferentes e linguagens de programação diferentes, compartilhando dos mesmos recursos do servidor original, como memória RAM, espaço em disco, dentre outros recursos, como pode ser visto na \autoref{fig:virtual-server}.

\begin{figure}
    \centering
    \caption{Utilização de Máquinas Virtuais como Ambiente de Testes}
    \label{fig:virtual-server}
    \includegraphics[width=0.6\linewidth]{./assets/images/virtual-server}
\end{figure}


