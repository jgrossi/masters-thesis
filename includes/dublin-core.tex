%!TEX root = ../masters.tex

\chapter{Elementos do padrão Dublin Core, versão 1.1.}
\label{attach:dublin-core}

\begin{flushleft}

    \begin{longtable}{|p{2cm}|p{2cm}|p{4cm}|p{6cm}|}
        \hline \textbf{Name} & \textbf{Label} & \textbf{Definition} & \textbf{Comment}\\ 
        \hline title & Title & A name given to the resource. & \\
        \hline creator & Creator & An entity primarily responsible for making the resource. & Examples of a Creator include a person, an organization, or a service.  Typically, the name of a Creator should be used to indicate the entity.\\
        \hline subject & Subject & The topic of the resource. & Typically, the subject will be represented using keywords, key phrases, or classification codes. Recommended best practice is to use a controlled vocabulary. To describe the spatial or temporal topic of the resource, use the Coverage element.\\
        \hline description & Description & An account of the resource. & Description may include but is not limited to: an abstract, a table of contents, a graphical representation, or a free-text account of the resource.\\
        \hline publisher & Publisher & An entity responsible for making the resource available. & Examples of a Publisher include a person, an organization, or a service.  Typically, the name of a Publisher should be used to indicate the entity.\\
        \hline contributor & Contributor & An entity responsible for making contributions to the resource. & Examples of a Contributor include a person, an organization, or a service. Typically, the name of a Contributor should be used to indicate the entity.\\
        \hline date & Date & A point or period of time associated with an event in the lifecycle of the resource. & Date may be used to express temporal information at any level of granularity.  Recommended best practice is to use an encoding scheme, such as the W3CDTF profile of ISO 8601 [W3CDTF].\\
        \hline type & Type & The nature or genre of the resource. & Recommended best practice is to use a controlled vocabulary such as the DCMI Type Vocabulary [DCTYPE]. To describe the file format, physical medium, or dimensions of the resource, use the Format element.\\
        \hline format & Format & The file format, physical medium, or dimensions of the resource. & Examples of dimensions include size and duration. Recommended best practice is to use a controlled vocabulary such as the list of Internet Media Types [MIME].\\
        \hline identifier & Identifier & An unambiguous reference to the resource within a given context. & Recommended best practice is to identify the resource by means of a string conforming to a formal identification system.\\
        \hline source & Source & A related resource from which the described resource is derived. & The described resource may be derived from the related resource in whole or in part.  Recommended best practice is to identify the related resource by means of a string conforming to a formal identification system.\\
        \hline language & Language & A language of the resource. & Recommended best practice is to use a controlled vocabulary such as RFC 4646 [RFC4646].\\
        \hline relation & Relation & A related resource. & Recommended best practice is to identify the related resource by means of a string conforming to a formal identification system.\\
        \hline coverage & Coverage & The spatial or temporal topic of the resource, the spatial applicability of the resource, or the jurisdiction under which the resource is relevant. & Spatial topic and spatial applicability may be a named place or a location specified by its geographic coordinates.  Temporal topic may be a named period, date, or date range.  A jurisdiction may be a named administrative entity or a geographic place to which the resource applies.  Recommended best practice is to use a controlled vocabulary such as the Thesaurus of Geographic Names [TGN].  Where appropriate, named places or time periods can be used in preference to numeric identifiers such as sets of coordinates or date ranges.\\
        \hline rights & Rights & Information about rights held in and over the resource. & Typically, rights information includes a statement about various property rights associated with the resource, including intellectual property rights.\\
        \hline
    \end{longtable}

\end{flushleft}
