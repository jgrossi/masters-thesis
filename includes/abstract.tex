%!TEX root = ../masters.tex

% resumo em português
\setlength{\absparsep}{18pt} % ajusta o espaçamento dos parágrafos do resumo
\begin{resumo}
 % Segundo a \citeonline[3.1-3.2]{NBR6028:2003}, o resumo deve ressaltar o
 % objetivo, o método, os resultados e as conclusões do documento. A ordem e a extensão
 % destes itens dependem do tipo de resumo (informativo ou indicativo) e do
 % tratamento que cada item recebe no documento original. O resumo deve ser
 % precedido da referência do documento, com exceção do resumo inserido no
 % próprio documento. (\ldots) As palavras-chave devem figurar logo abaixo do
 % resumo, antecedidas da expressão Palavras-chave:, separadas entre si por
 % ponto e finalizadas também por ponto.

 % \textbf{Palavras-chaves}: latex. abntex. editoração de texto.

São inúmeras as ferramentas para extração de metadados em artigos científicos, tendo cada uma sua particularidade, tecnologia e técnicas utilizadas. Porém, com a crescente produção científica e a grande variedade de editoras, eventos e congressos, um grande número de artigos permanece sem uma extração de metadados eficaz, o que dificulta a disseminação de conhecimento e principalmente a pesquisa eletrônica destes documentos.

Este trabalho realiza um teste com algumas ferramentas pré-selecionadas com um conjunto pré-determinado de artigos, que abrange diversas áreas do conhecimento, diversos eventos e formatos visuais diferentes. Estes testes são realizados em ambientes pré-configurados de acordo com a necessidade tecnológica de cada ferramenta, permitindo que todos os artigos tenham seus metadados extraídos por cada uma delas e seus resultados comparados individualmente. 

Desta forma, com base nos resultados apresentados, pode-se identificar o comportamento de cada uma das ferramentas perante à extração de metadados, suas falhas, onde são necessários ajustes e onde se obtém um maior sucesso na extração. Além disso, é apresentado também o um índice de confiabilidade, onde cada ferramenta recebe uma nota com base nos resultados obtidos na extração de metadados pela seleção de artigos realizada.

\textbf{Palavras-chaves}: artigos científicos, extração de metadados, extração de dados em artigos.

\end{resumo}

% resumo em inglês
\begin{resumo}[Abstract]
\begin{otherlanguage*}{english}

Currently we can find numerous tools to extract metadata from scientific papers, each one with its own particularity, technology and used techniques. However, with the increasing scientific production and the numerous publishers, events and conferences, a large part of papers still remain without an effective automated metadata extraction, hindering the knowledge dissemination and mainly the electronic search for these documents. 

The present work makes tests with pre selected tools over a set of scientific papers, covering different areas of knowledge and different layouts. These tests were made inside custom environments according the technologies each tool needs, allowing all papers to be tested and their metadata extracted, comparing results one by one. 

Thereby, according the presented results, it's possible to identify the behavior of each tool related to the metadata extraction, where it failed, where adjusts are needed and where it has success on the extraction. Moreover, it's also presented a reliability index, built from a grade given to each tool based on the metadata extraction results using the selected papers.

\textbf{Palavras-chaves}: scientific papers, metadata extraction, data extraction in scientific papers.

% \vspace{\onelineskip}

% \noindent 
% \textbf{Key-words}: latex. abntex. text editoration.
\end{otherlanguage*}
\end{resumo}

% resumo em francês 
% \begin{resumo}[Résumé]
%  \begin{otherlanguage*}{french}
%   Il s'agit d'un résumé en français.
 
%    \textbf{Mots-clés}: latex. abntex. publication de textes.
%  \end{otherlanguage*}
% \end{resumo}

% % resumo em espanhol
% \begin{resumo}[Resumen]
%  \begin{otherlanguage*}{spanish}
%    Este es el resumen en español.
  
%    \textbf{Palabras clave}: latex. abntex. publicación de textos.
%  \end{otherlanguage*}
% \end{resumo}
% ---