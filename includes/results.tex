%!TEX root = ../masters.tex

\chapter{Análise e Apresentação de Resultados}
\label{cha:results}

% Construção da teoria

% Falar sobre o corpus e as ferramentas (FEITO)

Com o Corpus totalmente definido e as ferramentas devidamente instaladas no ambiente de testes foram realizados diversos experimentos para que os resultados pudessem ser analisados e comparados numericamente.

% Falar das dificuldades encontradas na extração (FEITO)

Durante a extração dos metadados algumas observações puderam ser feitas tanto pela análise manual de cada resultado individual como também dos resultados em conjunto, tendo em vista os números apresentados pelas ferramentas utilizadas.

A ferramenta Cermine demonstrou-se de bem simples execução. Por se tratar de um arquivo em formato \texttt{.jar} (Java) em forma executável, a extração ocorreu de maneira fluida, com os dados de saída da ferramenta gravados em arquivos isolados para posterior comparação. Além disso, os resultados apresentados foram os mais completos, com utilização de diversas \emph{tags} XML que permitiram uma fácil manipulação dos dados, com uma grande riqueza de detalhes. O processo de extração dos metadados para a ferramenta Cermine foi o mais lento das 4 (quatro) ferramentas testadas, demorando entre 15 (quinze) e 20 (vinte) segundos para uma completa análise de cada artigo.

Já a ferramenta CiteSeer foi a que mais exigiu conhecimentos técnicos para que pudesse ser testada. Sua execução dependeu da instalação de diversos outros componentes e serviços de terceiros, o que contribuiu para um aumento da complexidade de seu uso. Um fato interessante é que a ferramenta utiliza de outras ferramentas para alguns processos específicos de extração, como é o caso da sessão de referências, onde utiliza a ferramenta ParsCit. Embora a ferramenta utilizada seja a mesma a forma de entrada de dados é diferenciada, implicando em resultados numericamente diferentes.

No caso da ferramenta CrossRef algumas particularidades devem ser mencionadas. Seus resultados de extração são apresentados de maneira muito básica, com campos muito genéricos e resultados pouco precisos, dificultando um pós-processamento dos dados. Os metadados ``autores'', ``e-mails'' e ``resumo'' não puderam ser extraídos. A versão atual de desenvolvimento da ferramenta não permite uma separação de dados muito específica, agrupando diversas informações em tags chamadas ``sections''. Estas tags possuem informações textuais gerais, não sendo possível serem filtradas com a utilização da própria ferramenta. Portanto, para a ferramenta CrossRef somente os metadados ``título'' e ``referências'' foram extraídos e considerados. Os resultados para a extração das referências também merecem considerações, por serem apresentados de maneira muito genérica, em uma única tag XML, sendo impossível separar título e autor dentro do conteúdo retornado.

A ferramenta ParsCit também foi utilizada sem maiores dificuldades. Em virtude de sua particularidade de processar apenas dados de entrada em formato texto ou XML, conforme sugerido pelos desenvolvedores, foi utilizada a ferramenta de linha de comando \texttt{pdftotext} (disponível em ambiente Linux) para conversão dos artigos em \texttt{.pdf} em arquivos \texttt{.txt}, permitindo que a ferramenta fosse utilizada conforme recomendações. Esta conversão foi feita em tempo de execução e os resultados coletados e gravados com sucesso.

De modo geral, exceto pela ferramenta CrossRef as demais ferramentas tiveram um processo de extração bem eficaz visualmente e dentro do esperado, em virtude da grande diferenciação visual testada com o Corpus selecionado. 

% Falar das particularidades nos algoritmos de comparação (FEITO)

No que diz respeito à comparação dos resultados foi necessária uma padronização dos dados para que as quatro ferramentas pudessem ser testadas de maneira uniforme. Em virtude de apresentar resultados bastante detalhados, a ferramenta Cermine permitiu que os autores das referências fossem retornados seguindo a forma ``primeiro nome'' e em seguida ``sobrenome''. Já as demais ferramentas não apresentaram os resultados com tanta flexibilidade, variando em alguns momentos a ordem e disposição do nome dos autores. Assim, foi necessário um pré-processamento computacional a fim de manter, quando possível, o primeiro nome antes do sobrenome, tornando as comparações mais padronizadas e justas. Este pré-processamento foi realizado para todas as ferramentas testadas.

Já para a extração do metadado ``e-mails'', algumas ferramentas extraíram mais informações em conjunto, como foi o caso de algumas poucas extrações realizadas pela ferramenta Cermine. Em um destes casos a ferramenta retornou como e-mail o seguinte conteúdo: \texttt{Email: mvpein@yahoo.com}. Assim, sempre visando uma justa comparação entre as ferramentas foi realizada uma análise em todos os resultados deste metadado para que somente pudessem ser comparados endereços de e-mails, o que tornou o processo bem simplificado e correto. Os endereços de e-mail foram filtrados com a utilização de expressão regular (\autoref{sssec:regular-expressions}) alcançando um conjunto homogêneo de dados comparados.

Os demais metadados foram comparados sem problemas. O metadado ``título'' foi utilizado sem sua pontuação final, retirando antes da comparação qualquer caractere passível de erros como: asteriscos, pontos finais e espaços em branco. O resultado das extrações dos título foi feito seguindo a lógica anteriormente apresentada, comparando a similaridade entre os dois textos através do uso da função \texttt{similar\_text} da linguagem de programação PHP, que apresenta como resultado um valor numérico representando o percentual de similaridade. Esta mesma lógica descrita foi aplicada para o metadado ``resumo''.

Os nomes dos autores foram comparados seguindo a mesma lógica do metadado ``título'', porém levando em consideração a ordem de apresentação e extração dos mesmos. Sendo assim, além de verificar a similaridade entre os nomes os testes levaram em consideração a ordem de apresentação dos resultados de cada ferramenta.

% Falar também da questão dos nomes dos autores com acentos, no caso de autores russos, croatas e latinos. Algumas ferramentas retornaram os resultados sem acento o que implica em uma comparação menos precisa. (FEITO)

Dentro do Corpus escolhido diversos nomes de autores continham acentos e caracteres característicos de seu idioma de origem, como é o caso da autora polonesa ``Anna Białk-Bielińska''. Em virtude desta questão, as ferramentas se comportaram de maneiras distintas. Algumas conseguiram extrair os nomes como no artigo original, porém, outras substituíram caracteres como ``ń'' por apenas ``n'', ou ainda ``n´''. Algumas ferramentas simplesmente desprezaram estes caracteres.

No caso específico do metadado ``e-mail'' a comparação foi realizada com base na identificação correta ou não do endereço eletrônico. Neste caso não foi considerada a porcentagem de similaridade entre resultados, ou seja, o endereço foi corretamente identificado ou não. Para estes resultados foram utilizados os inteiros 0 (zero) para a extração ineficiente e 100 (cem) para a extração eficiente.

% Falar que considerei apenas os artigos que possuem email. Os que não tinha o cálculo da média desconsidera os resultados, marcando-os como -1. (FEITO)

Uma grande parte dos artigos utilizados no Corpus deste trabalho não continha informações de e-mail dos autores. Desta forma, as extrações destes documentos foram desconsideradas, permitindo que as ferramentas tivessem seus resultados avaliados apenas para as extrações realmente computadas, valorizando ainda mais o trabalho de cada uma.

% Sobre a comparação dos Títulos (FEITO)

Já para a comparação das referências foram utilizadas duas informações: o título e o nome dos autores. Para o caso do título das referências, a lógica utilizada foi a mesma para o metadado ``título'', utilizando-se de um valor percentual representando a similaridade entre os dados comparados. Já para o nome dos autores, a lógica seguiu a mesma do metadado ``autores'', onde foram consideradas tanto a similaridade textual como também a ordem de apresentação. Deste modo, a extração de cada referência considerou um peso de 60\% do resultado para o título e 40\% para os nomes dos autores, chegando em um numero final que representasse o resultado da extração de cada referência comparada.

% Resultados das comparações (FEITO)

Com os dados de cada extração armazenados a comparação foi feita de maneira automática levando em consideração todos os pontos apresentados acima. Para cada subárea do conhecimento foi realizada uma comparação, registrando o resultado consolidado para cada artigo extraído, bem como a média aritmética dos resultados daquela subárea em específico. Portanto, para cada ferramenta e subárea, foi registrado um valor médio dos resultados.

Posteriormente foi feita a coleta destes dados separados por subáreas, porém consolidando-os para cada ferramenta. Assim, foi calculada a média aritmética dos resultados de cada ferramenta para todas as subáreas analisadas, chegando então a uma nota final em cada metadado extraído, possibilitando então o cálculo do ``Índice de Confiabilidade'' (\autoref{ssec:confiability-index}) para cada ferramenta.

% Trabalhar as evidências de que sua hipótese é verdadeira

\section{Resultados}
\label{sec:results}

% Apresentar dados, testes, provas, estudos de caso, etc

Conforme esperado os resultados foram coletados de maneira individualizada - para cada artigo - e consolidados de maneira geral para cada ferramenta e metadado. Os resultados apresentados por área do conhecimento estão presentes em 4 (quatro) tabelas, separadas por cada uma das ferramentas. 

Os resultados da ferramenta Cermine estão presentes na \autoref{tab:results-cermine}. Os resultados da CiteSeer estão na \autoref{tab:results-citeseer}. Os resultado da ferramenta CrossRef na \autoref{tab:results-crossref} e da ParsCit na \autoref{tab:results-parscit}. Todas as tabelas mostram o percentual de acerto separados por subárea do conhecimento e por metadados, representados pelas colunas Tit. (Título),  Aut. (Autores), Ema. (E-mails), Res. (Resumo) e Ref. (Referências).

\begin{table}
    \caption{Resultados da ferramenta Cermine por subárea do conhecimento.}
    \begin{center}
        \begin{tabular}{|p{6cm}|C{1.2cm}|C{1.2cm}|C{1.2cm}|C{1.2cm}|C{1.2cm}|}
            \hline 
            \textbf{Subárea do Conhecimento} & \textbf{Tit.} & \textbf{Aut.} & \textbf{Ema.} & \textbf{Res.} & \textbf{Ref.} \\ \hline 
            Arquitetura e Urbanismo & 100 & 58.75 & 16.67 & 99.01 & 82.67 \\ \hline
Ciência da Computação & 88.27 & 71.87 & 21.43 & 98.83 & 77.25 \\ \hline
Ciência da Informação & 76.55 & 61.90 & 28.57 & 78.02 & 53.81 \\ \hline
Ciências Biológicas (Genética) & 91.58 & 81.00 & 50.00 & 84.72 & 96.11 \\ \hline
Ciências Biológicas (Zoologia) & 99.78 & 73.16 & 42.86 & 84.74 & 72.28 \\ \hline
Enfermagem & 99.77 & 39.38 & 16.67 & 98.09 & 81.69 \\ \hline
Engenharia Civil & 71.43 & 76.34 & 37.50 & 94.18 & 56.23 \\ \hline
Engenharia Mecânica & 99.45 & 75.97 & 58.33 & 77.97 & 82.87 \\ \hline
Fonoaudiologia & 100 & 77.75 & 71.43 & 98.13 & 80.05 \\ \hline
Geologia & 99.54 & 100 & 66.67 & 53.66 & 64.03 \\ \hline
História & 99.20 & 89.29 & 50.00 & 65.59 & 53.40 \\ \hline
Letras & 88.01 & 99.50 & 42.86 & 82.10 & 86.74 \\ \hline
Medicina Veterinária & 85.71 & 91.11 & 85.71 & 98.77 & 80.05 \\ \hline
Música & 99.03 & 90.61 & 66.67 & 95.47 & 68.50 \\ \hline
Psicologia & 88.46 & 63.96 & 47.62 & 92.53 & 63.25 \\ \hline
Zootecnia & 49.95 & 70.82 & 42.86 & 87.40 & 81.99 \\ \hline
\textbf{Média Geral} & 89.80 & 76.34 & 46.62 & 86.83 & 73.81 \\ \hline

        \end{tabular}
    \end{center}
    \label{tab:results-cermine}
\end{table}

\begin{table}
    \caption{Resultados da ferramenta CiteSeer por subárea do conhecimento.}
    \begin{center}
        \begin{tabular}{|p{6cm}|C{1.2cm}|C{1.2cm}|C{1.2cm}|C{1.2cm}|C{1.2cm}|}
            \hline 
            \textbf{Subárea do Conhecimento} & \textbf{Tit.} & \textbf{Aut.} & \textbf{Ema.} & \textbf{Res.} & \textbf{Ref.} \\ \hline 
            Arquitetura e Urbanismo & 100 & 96.89 & 0 & 97.43 & 70.95 \\ \hline
Ciência da Computação & 100 & 83.75 & 23.81 & 99.81 & 71.79 \\ \hline
Ciência da Informação & 84.44 & 99.50 & 0 & 74.12 & 55.56 \\ \hline
Ciências Biológicas (Genética) & 80.92 & 83.15 & 28.57 & 60.63 & 25.15 \\ \hline
Ciências Biológicas (Zoologia) & 57.14 & 64.12 & 0 & 71.14 & 70.10 \\ \hline
Enfermagem & 71.43 & 52.82 & 0 & 70.31 & 34.67 \\ \hline
Engenharia Civil & 97.81 & 62.42 & 0 & 71.18 & 35.61 \\ \hline
Engenharia Mecânica & 71.11 & 46.00 & 0 & 71.36 & 63.18 \\ \hline
Fonoaudiologia & 100 & 61.14 & 0 & 94.68 & 61.85 \\ \hline
Geologia & 73.77 & 34.69 & 0 & 42.79 & 57.13 \\ \hline
História & 99.53 & 71.09 & 0 & 65.26 & 63.81 \\ \hline
Letras & 99.57 & 85.73 & 0 & 75.78 & 58.82 \\ \hline
Medicina Veterinária & 85.71 & 86.38 & 0 & 98.88 & 63.53 \\ \hline
Música & 49.02 & 56.87 & 0 & 54.28 & 54.55 \\ \hline
Psicologia & 94.93 & 83.85 & 14.29 & 88.87 & 66.75 \\ \hline
Zootecnia & 71.43 & 82.41 & 0 & 76.39 & 22.59 \\ \hline
\textbf{Média Geral} & 83.55 & 71.93 & 4.17 & 75.81 & 54.75 \\ \hline

        \end{tabular}
    \end{center}
    \label{tab:results-citeseer}
\end{table}

\begin{table}
    \caption{Resultados da ferramenta CrossRef por subárea do conhecimento.}
    \begin{center}
        \begin{tabular}{|p{6cm}|C{1.2cm}|C{1.2cm}|C{1.2cm}|C{1.2cm}|C{1.2cm}|}
            \hline 
            \textbf{Subárea do Conhecimento} & \textbf{Tit.} & \textbf{Aut.} & \textbf{Ema.} & \textbf{Res.} & \textbf{Ref.} \\ \hline 
            Arquitetura e Urbanismo & 72.68 & 0 & 0 & 0 & 22.79 \\ \hline
Ciência da Computação & 64.19 & 0 & 0 & 0 & 14.64 \\ \hline
Ciência da Informação & 32.32 & 0 & 0 & 0 & 8.14 \\ \hline
Ciências Biológicas (Genética) & 47.05 & 0 & 0 & 0 & 14.62 \\ \hline
Ciências Biológicas (Zoologia) & 70.70 & 0 & 0 & 0 & 32.72 \\ \hline
Enfermagem & 55.96 & 0 & 0 & 0 & 10.29 \\ \hline
Engenharia Civil & 74.70 & 0 & 0 & 0 & 12.21 \\ \hline
Engenharia Mecânica & 89.27 & 0 & 0 & 0 & 27.50 \\ \hline
Fonoaudiologia & 71.43 & 0 & 0 & 0 & 13.92 \\ \hline
Geologia & 97.62 & 0 & 0 & 0 & 15.72 \\ \hline
História & 64.08 & 0 & 0 & 0 & 16.11 \\ \hline
Letras & 75.66 & 0 & 0 & 0 & 32.58 \\ \hline
Medicina Veterinária & 49.66 & 0 & 0 & 0 & 23.09 \\ \hline
Música & 84.63 & 0 & 0 & 0 & 28.16 \\ \hline
Psicologia & 82.92 & 0 & 0 & 0 & 23.19 \\ \hline
Zootecnia & 32.05 & 0 & 0 & 0 & 25.21 \\ \hline
\textbf{Média Geral} & 66.56 & 0 & 0 & 0 & 20.06 \\ \hline

        \end{tabular}
    \end{center}
    \label{tab:results-crossref}
\end{table}

\begin{table}
    \caption{Resultados da ferramenta ParsCit por subárea do conhecimento.}
    \begin{center}
        \begin{tabular}{|p{6cm}|C{1.2cm}|C{1.2cm}|C{1.2cm}|C{1.2cm}|C{1.2cm}|}
            \hline 
            \textbf{Subárea do Conhecimento} & \textbf{Tit.} & \textbf{Aut.} & \textbf{Ema.} & \textbf{Res.} & \textbf{Ref.} \\ \hline 
            Arquitetura e Urbanismo & 0 & 17.14 & 0 & 73.23 & 51.62 \\ \hline
Ciência da Computação & 37.54 & 58.36 & 47.62 & 74.82 & 69.72 \\ \hline
Ciência da Informação & 32.30 & 31.51 & 28.57 & 59.60 & 50.09 \\ \hline
Ciências Biológicas (Genética) & 8.97 & 1.17 & 0 & 40.91 & 39.76 \\ \hline
Ciências Biológicas (Zoologia) & 0 & 0 & 0 & 69.98 & 54.61 \\ \hline
Enfermagem & 11.06 & 14.29 & 0 & 65.65 & 37.24 \\ \hline
Engenharia Civil & 11.48 & 15.64 & 37.50 & 56.61 & 37.97 \\ \hline
Engenharia Mecânica & 14.29 & 23.23 & 22.22 & 55.93 & 55.34 \\ \hline
Fonoaudiologia & 5.70 & 0.89 & 0 & 27.74 & 62.52 \\ \hline
Geologia & 14.29 & 14.29 & 16.67 & 68.42 & 55.15 \\ \hline
História & 5.62 & 14.29 & 0 & 59.33 & 60.17 \\ \hline
Letras & 24.42 & 42.21 & 21.43 & 68.53 & 54.00 \\ \hline
Medicina Veterinária & 14.29 & 13.82 & 14.29 & 36.57 & 53.31 \\ \hline
Música & 34.24 & 42.86 & 0 & 81.39 & 55.73 \\ \hline
Psicologia & 14.29 & 14.29 & 14.29 & 71.14 & 68.84 \\ \hline
Zootecnia & 14.29 & 5.88 & 0 & 78.60 & 51.21 \\ \hline
\hline \textbf{Média Geral} & 15.17 & 19.37 & 12.66 & 61.78 & 53.58 \\ \hline

        \end{tabular}
    \end{center}
    \label{tab:results-parscit}
\end{table}

% Conceitos criados pelo autor (FEITO)

Para que os resultados pudessem ser melhores interpretados foi calculado o ``Índice de Confiabilidade'' de cada ferramenta, detalhado no capítulo de Metodologia (\autoref{ssec:confiability-index}). Para calcular este índice foram utilizadas as médias dos resultados de extração de todas as subáreas, com os devidos pesos para cada metadado, obtendo-se então uma nota geral para cada ferramenta. Os resultados calculados para este índice estão presentes na \autoref{tab:conf-index-results}.

\begin{table}
    \caption{Índice de Confiabilidade de cada ferramenta}
    \begin{center}
        \begin{tabular}{|p{3cm}|C{3cm}|}
            \hline 
            \textbf{Ferramenta} & \textbf{Resultado} \\ \hline 
            Cermine & 79.81 \\ \hline 
CiteSeer & 68.00 \\ \hline 
CrossRef & 24.30 \\ \hline 
ParsCit & 33.27 \\ \hline 

        \end{tabular}
    \end{center}
    \label{tab:conf-index-results}
\end{table}

De posse do ``Índice de Confiabilidade'' de cada ferramenta, conforme previsto na \autoref{ssec:confiability-index}, cada uma foi classificada de acordo com seus resultados de extração. Estes resultados e suas respectivas classificações estão presentes na \autoref{tab:tool-classification}.

\begin{table}
    \caption{Classificação de cada ferramenta.}
    \begin{center}
        \begin{tabular}{|p{3cm}|C{5.5cm}|p{4cm}|}
            \hline 
            \textbf{Ferramenta} & \textbf{Índice de Confiabilidade} & \textbf{Classificação} \\ \hline 
            Cermine & 79.81 & Satisfatória \\ \hline 
CiteSeer & 68.00 & Satisfatória \\ \hline 
CrossRef & 24.30 & Insatisfatória \\ \hline 
ParsCit & 33.27 & Insatisfatória \\ \hline 

        \end{tabular}
    \end{center}
    \label{tab:tool-classification}
\end{table}

\section{Ambiente de Testes}

Para a realização das extrações e das comparações foi criado um ambiente de testes contendo todas as tecnologias necessárias para que as ferramentas pudessem ser executadas dentro do esperado. Desta maneira, foi utilizado um servidor virtual com a seguinte configuração:

\begin{itemize}
    \item Sistema Operacional Linux Ubuntu 14.04 64 Bits
    \item 2GB de Memória RAM
    \item 20GB de Espaço em Disco
\end{itemize}

As tecnologias utilizadas foram instaladas de acordo com as recomendações de cada ferramenta, com suas dependências e necessidades de cada linguagem de programação. Foram instaladas as seguintes linguagens/bibliotecas, separadas de acordo com cada ferramenta:

\begin{itemize}
    \item \textbf{Cermine:} Java OpenJDK Runtime Environment 1.7.0\_79
    \item \textbf{CiteSeer:} Python 2.7.6, GROBID \url{https://github.com/kermitt2/grobid}, PDFBox \url{http://pdfbox.apache.org/}, PDF Classifier Jar, Java SE Environment (Maven). 
    \item \textbf{CrossRef:} Ruby 2.1.2p95, RubyGem \texttt{pdf-extract} 0.0.1 e \texttt{pdf-reader} 1.3.2.
    \item \textbf{ParsCit:} Perl 5.18.2, G++ Compiler e CRF++ 0.51. Diversas outras dependências da linguagem Perl foram também instaladas: Class::Struct, Getopt::Long, Getopt::Std, File::Basename, File::Spec, FindBin, HTML::Entities, IO::File, POSIX, XML::Parser, XML::Twig, XML::Writer e XML::Writer::String.
\end{itemize}

